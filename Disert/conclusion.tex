% -*-coding: utf-8;-*-
% Заключение

\begin{enumerate}

\item Разработана и успешно опробована в ряде имитационных
  экспериментов методика замены линейных П, ПИ, ПИД регуляторов на
  нейросетевой.  В рамках методики решаются актуальные вопросы выбора
  архитектуры нейронной сети регулятора и параметров экспериментальных
  выборок и алгоритма обучения.  Исследуется влияние указанных
  аспектов, а также вида и длины пробных сигналов на качество имитации
  традиционного регулятора нейросетевым.  Показывается применимость
  методики для управления существенно нелинейным объектом.
\item Предложен алгоритм синтеза нейросетевого регулятора,
  минимизирующий среднеквадратическую ошибку управления.  Алгоритм
  использует нейросетевую инверсию по модели предсказания объекта и
  рассматривается в случае стохастических сигналов уставки и помехи.
  Благодаря сходству постановок задачи управления проводится аналогия
  нейросетевого оптимального и винеровского оптимального регуляторов.
  Отмечается преимущество нейросетевого подхода как с точки зрения
  качества управления и робастности, так и с точки зрения удобства
  синтеза в инженерной практике.
\item Разработаны два нейросетевых алгоритма управления нестационарным
  объектом: с постоянной адаптацией регулятора и модели, и с
  адаптацией по обнаружению разладки.  В последнем случае для
  обнаружения разладки используется алгоритм кумулятивных сумм.
  Исследован вопрос накопления обучающей выборки для подстройки модели
  объекта.  Проведено сравнение обоих алгоритмов.  Отмечено, что
  алгоритм с обнаружением разладки обеспечивает большую экономичность
  и устойчивость управления, чем метод с постоянной адаптацией.
\item Успешно решена задача нейросетевого управления автономным
  мобильным роботом при движении на неподвижный маяк.  При этом
  использовались предложенные методики по замене П регулятора на
  нейросетевой и синтеза нейросетевого оптимального регулятора.
\item Развитые методы нейросетевого управления реализованы в модульном
  интерактивном программном комплексе, обеспечивающем полную среду для
  изучения и сопоставления нейросетевых алгоритмов в системах
  автоматического управления.  Данный комплекс адаптирован для
  использования в учебном процессе в качестве базового для проведения
  лабораторных и исследовательских работ студентами.

\end{enumerate}
