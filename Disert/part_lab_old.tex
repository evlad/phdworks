% -*-coding: utf-8;-*-



\hrule
%%%%%%%%%%%%%%%%%%%%%%%%%%%%%%%%%%%%%%%%%%%%%%%%%%%%%%%%%%%%%%%%%%%%%%

\subsection{Мотивация}

%  1.1. Важность практических занятий в освоении НС

Искусственные нейронные сети представляют собой относительно новый
мощный и многогранный инструмент для решения разнообразных инженерных
и научно-исследовательских задач.  Способ применения нейронных сетей
достаточно специфичен, так как он основан не на аналитическом
формализме, являющемся результатом работы человеческого интеллекта, а
на машинном обучении --- неявном извлечении взаимосвязей из
представленных данных.  Эта особенность отличает нейросетевые методы
от традиционных, в том числе и в задачах автоматического управления.

Аналитические методы давно и прочно укоренились в учебных курсах на
инженерных специальностях.  Для освоения аппарата этих методов
студентам предлагается большое количество разнообразных упражнений,
суть которых сводится к аналитическому решению задачи, то есть, к
выводу формальным способом решения в общем виде.  Только в том случае,
если математический аппарат не позволяет получить точное решение в
аналитической форме, используются приближенные способы решения, в том
числе, с помощью численных методов.  Однако в любом случае вид
решения, а значит и его свойства, определяются человеком.

Нейросетевые методы подразумевают участие человека только в
обеспечении процесса обучения, а собственно решение формируется
нейросетью само.  Его вид и свойства изначально неизвестны.  Более
того, после успешного завершения обучения нейронной сетью, отличной от
тривиальной, человеку практически невозможно понять, какие решения
приняла сеть и как это повлияло на достижение поставленной цели.
Извлечение знаний из обученной нейронной сети представляет собой
отдельную проблему.

Причину этого следует искать в том, что существующая на настоящий
момент теория искусственных нейронных сетей неполна.  В частности, не
существует математически доказанных конструктивных методов,
обеспечивающих решение базовой задачи: интерполяции произвольной
функции нейросетью.  Вопросы информационной ёмкости нейронной сети
решены только для небольшого класса сетей с линейными нейронами,
бесперспективность которого была показана ещё в 1960-х годах Мински
(???).  Архитектура и принципы функционирования нейронных сетей разных
типов чрезвычайно разнообразны, а математический аппарат, используемый
в нейросетевых методах решения прикладных задач, позаимствован из
разных областей прикладной математики.

Возможность применения нейронных сетей основывается только на теореме
(???) о существовании решения интерполяции произвольной функции
многослойным нелинейным персептроном.  В отсутствии теоретически
обоснованного конструктивного метода настройки нейронной сети широкое
распространение получили различные численные методы, близкие к
нелинейной оптимизации, а также разнообразные эвристические подходы.

Изложенные особенности свидетельствуют об особой важности практических
работ в учебном курсе по нейросетевым методам.  Только они дают
возможность студентам увидеть, как происходит процесс обучения и как
функционирует обученная нейронная сеть в рамках решения конкретной
прикладной задачи.  Это невозможно до конца формализовать на лекциях и
изложить в учебных пособиях.  Фактически, совокупность архитектуры НС,
метода обучения с его параметрами, а также обучающих данных дает
уникальное решение, выражающееся в наборе весовых коэффициентов НС.

%  1.2. Почему новый пакет?

Обычно для практических работ по курсу искусственных нейронных сетей
используется тот или иной универсальный (Statistica, MatLab, Octave)
или специализированный нейросетевой (Stuttgart Neural Network
Simulator, Neural Lab, Trajan) программный пакет.  Для большинства
типовых задач, решаемых с помощью НС (распознавание образов,
ассоциативная память, кластеризация, предсказание и т.п.),
возможностей перечисленных пакетов вполне достаточно.  К ним
относятся:

\begin{itemize}
\item Задание архитектуры нейросети и метода её обучения
\item Задание обучающего множества
\item Задание параметров обучения
\item Обучение нейронной сети
\item Анализ качества работы обученной нейронной сети
\end{itemize}

Однако применение НС в задачах управления требует дополнительно
наличия многих функций, отсутствующих в пакетах нейросетевого
моделирования:

\begin{itemize}
\item Задание вида и параметров регулятора и объекта управления
\item Задание входных сигналов - уставки и помехи
\item Съем данных из различных точек контура управления для
  визуализации и обучения НС
\item Моделирование САУ
\item Сравнение и анализ качества работы САУ с различными
  компонентами, в том числе, с нейросетевыми
\end{itemize}

В универсальных пакетах (класса MatLab) реализация перечисленных
функций требует достаточно серьезного программирования как
вычислительных, так и интерактивных и графических функций.  В то же
время, получающаяся программа обладала бы ограниченным
быстродействием, так как должна быть написана на интерпретируемом
языке программирования.  Вопросы быстродействия в классе задач
автоматического управления достаточно важны, так как моделирование и
обучение нейронных сетей производится на длинных временных рядах
(порядка $10^4 ... 10^6$ отсчетов).

Ещё одним классом программ, разрабатываемым в учебных целях, являются
демонстрационные программы чаще всего размещаемые с Интернете и
доступные для online-использования (??? привести перечень ссылок).
Такие программы обычно демонстрируют какой-то конкретный алгоритм
управления на конкретном объекте, например, управление обратным
маятником, поэтому рядом с online-интерфейсом программы расположено
описание алгоритма управления, системы или ссылки на такую информацию.
Для проведения моделирования пользователю достаточно запустить систему
моделирования.  Иногда имеется возможность задать некоторые параметры
системы и возмущающее воздействие.  Для наглядной демонстрации работы
САУ используется динамическая графика и, в целом, оформление сделано
весьма эффектно.  К достоинствам online-программ относится то, что их
не надо устанавливать на компьютер, они не потребляют компьютерных
ресурсов пользователя (за исключением web-браузера, используемого для
доступа на web-страницу программы) и доступны везде и всегда где есть
доступ в Интернет.

В то же время, у подобных программ, часто выглядящих как рекламный
ролик алгоритма управления, есть свои недостатки:
\begin{itemize}

\item узкая специализация на одной прикладной задаче, причем часто
  даже с фиксированными параметрами;

\item презентационная направленность, скрывающая от пользователя
  ``скучные'' элементы алгоритма и демонстрирующие только результат;

\item обычно отсутствует возможность сравнения с иными алгоритмами
  управления.

\end{itemize}

Таким образом, демонстрационные online-программы могут быть
использованы в учебном процессе очень ограниченно --- для иллюстрации
работы конкретных управляющих алгоритмов в конкретных задачах.  Для
практических занятий, подразумевающих деятельное участие студентов,
они совершенно непригодны.

Представляется актуальным разработать интерактивный пакет программ,
позволяющий решать задачи нейросетевого управления и сопоставлять
нейросетевые подходы с традиционными.  На базе такого пакета можно
создать курс практических занятий для студентов инженерных
специальностей, изучающих нейросетевые методы вообще и применение
нейронных сетей в системах автоматического управления в частности.

Расширение лабораторной базы учебного процесса позволит укрепить
знания, получаемые студентами на лекциях, практическим опытом решения
учебных задач.  

При определенной гибкости настроек подобный программный комплекс был
бы полезен и для исследовательских проектов, позволяя быстро проводить
моделирование проектируемой САУ и оценивать возможности использования
в ней нейросетевого управления.

%  1.3. Состав лабораторных работ


%  1.4. Дидактические аспекты


Совершенствование лабораторной базы.  Предлагается создать лабораторные работы.

Программный комплекс.
Задачи, возложенные на этот комплекс.

\subsection{Структура комплекса}

\subsection{Функциональная декомпозиция}

Функционально пакет должен обеспечивать решение следующих основных
задач:
\begin{enumerate}
\item Моделирование САУ с возможностью гибкого задания вида уставки,
  помехи, регулятора и объекта управления.  Необходимо предусмотреть
  возможность моделирования нестационарных и нелинейных объектов
  управления, а также использование нейронных сетей.
\item Создание и обучение нейронных сетей как в контуре, так и вне
  контура САУ.
\end{enumerate}

\begin{itemize}
\item Задание архитектуры нейросети и метода её обучения
\item Задание обучающего множества
\item Задание параметров обучения
\item Обучение нейронной сети
\item Анализ качества работы обученной нейронной сети
\end{itemize}

Однако применение НС в задачах управления требует дополнительно
наличия многих функций, отсутствующих в пакетах нейросетевого
моделирования:

\begin{itemize}
\item Задание вида и параметров регулятора и объекта управления
\item Задание входных сигналов - уставки и помехи
\item Съем данных из различных точек контура управления для
  визуализации и обучения НС
\item Моделирование САУ
\item Сравнение и анализ качества работы САУ с различными
  компонентами, в том числе, с нейросетевыми
\end{itemize}


\subsection{Принципы взаимодействия}




%\subsection{Дидактические цели}
%Что должен знать и уметь студент по результатам выполнения лабораторных работ.
%\subsection{План методички}
%\begin{itemize}
%\item Введение: цель работы
%\item Теоретическая часть - взять из 2, 3, 4 глав
%\item Задание на выполнение работы
%\item Инструкция по проведению работы (описание программного комплекса)
%\item Представление результатов
%\item Контрольные вопросы
%\end{itemize}

%\section{Методическая база}
%\subsection{Примеры отчетов - в приложении}


\begin{table}[ht]
  \centering
  \caption{Перечень реализованных в пакете нелинейных звеньев.}
  \label{tabl:nonlinear_functions}
  \begin{tabular}{|c|l|l|}
    \hline
    Имя звена & Описание \\
    \hline
    \tt saturat & Зона насыщения: \\
    & $f(x)=\left\{
      \begin{array}{rl}
        KL,  & x \ge L/K \\
        Kx,  & -L/K < x < L/K \\
        -KL,  & x \le -L/K \\
      \end{array}\right.$ \\
    \hline
    \tt nosense & Область нечувствительности: \\
    & $f(x)=\left\{
      \begin{array}{rl}
        Kx,  & x > H \\
        0,   & -H \le x \le H \\
        Kx, & x < -H \\
      \end{array}\right.$ \\
    \hline
    \tt luft & Люфт \\
    \hline
    \tt sine & Синус \\
    \hline
  \end{tabular}
\end{table}

