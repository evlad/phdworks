% -*-coding: utf-8;-*-
%%%%%%%%%%%%%%%%%%%%%%%%%%%%%%%%%%%%%%%%%%%%%%%%%%%%%%%%%%%%%%%%%%%%%%%%%%%%%%

\begin{enumerate}
\item Разработан интерактивный программный комплекс для моделирования
  одноконтурных систем автоматического управления в дискретном времени
  с возможностью использования линейных, нелинейных и нестационарных
  объектов.  В пакете имеются развитые средства для синтеза
  нейросетевого регулятора и нейросетевой модели объекта управления
  для стационарного и нестационарного объектов, а также возможность
  моделирования контура управления с включенными нейросетевыми
  элементами.
\item Предложена методика использования программного комплекса в
  учебном процессе на инженерных специальностях высших учебных
  заведений для практических занятий и научных исследований в курсах
  по нейронным сетям и по системам управления.
\item Разработан курс из трех лабораторных работ по применению
  нейронных сетей в системах управления, включающий освоение
  студентами методов синтеза нейросетевых систем управления
  стационарными и нестационарными объектами, а также сопоставление
  нейросетевого, винеровского оптимального и ПИД регулятора в
  различных условиях.
\end{enumerate}
