% -*-coding: utf-8;-*-
%%%%%%%%%%%%%%%%%%%%%%%%%%%%%%%%%%%%%%%%%%%%%%%%%%%%%%%%%%%%%%%%%%%%%%%%%%%%%%
%\section{Описание лабораторных работ}

%%%%%%%%%%%%%%%%%%%%%%%%%%
% Лабораторная работа №1 %
%%%%%%%%%%%%%%%%%%%%%%%%%%
\subsection{Синтез нейросетевого регулятора}

\subsubsection{Цель работы}

Лабораторная работа посвящена задаче синтеза нейросетевого
оптимального регулятора.  В ней изучаются принципы применения
нейронных сетей в качестве регулятора и модели объекта управления.
Исследуется влияние архитектуры нейронной сети на скорость и качество
обучения.  Полученный нейросетевой оптимальный регулятор используется
в последующих лабораторных работах.

\subsubsection{Постановка задачи}

Имеется система управления с обратной связью.  У объекта управления
один управляющий вход и один наблюдаемый выход.  Известно, что в
канале наблюдения имеется случайная помеха.  Дана стохастическая
модель сигнала уставки.  Необходимо заменить регулятор на нейросетевой
с минимизацией квадрата ошибки управления.

Для простоты понимания задачи и в целях последующего сравнения с
линейными регуляторами в качестве объекта управления и регулятора
берутся линейные звенья, хотя методика синтеза никак не ограничивает в
выборе класса управляемого объекта и заменяемого регулятора.

В процессе выполнения промежуточных этапов синтеза исследуется влияние
архитектуры нейронной сети (количество слоев, распределение нейронов в
них) на скорость обучения и достигаемое качество.

\subsubsection{Варианты}

Различные варианты данной работы можно обеспечить, задавая разные
параметры объекта управления, уставки и помехи.

\subsubsection{План работы}

Работа состоит из последовательного выполнения следуюих подзадач:
\begin{enumerate}
\item Моделирование САУ с линейным регулятором в контуре с целью сбора
  данных для обучения нейронных сетей.
\item Задание архитектуры нейросетевой модели объекта управления.
\item Обучение нейросетевой модели предсказанию наблюдаемого выхода
  объекта управления.
\item Повторить два предыдущих пункта для разных архитектур нейронных
  сетей.  Выбрать наилучший по минимуму ошибки предсказания на
  тестовой выборке.
\item Задание архитектуры нейросетевого регулятора.
\item Обучение нейросетевого регулятора функционированию подобно
  исходному вне контура управления.
\item Повторить два предыдущих пункта для разных архитектур нейронных
  сетей.  Выбрать наилучший по минимуму ошибки имитации на тестовой
  выборке.
\item Замена исходного регулятора нейросетевым в контуре управления и
  включение контура его адаптации с помощью нейросетевой модели.
\item Моделирование САУ с обученным нейросетевым регулятором в контуре
  с целью сравнения его с исходным.
\end{enumerate}

\subsubsection{Пример отчета}


%%%%%%%%%%%%%%%%%%%%%%%%%%
% Лабораторная работа №2 %
%%%%%%%%%%%%%%%%%%%%%%%%%%
\subsection{Сравнение нейросетевого, винеровского оптимального и ПИД регулятора}

\subsubsection{Цель работы}

Работа посвящена исследованию свойств нейросетевого регулятора в
сравнении с линейными регуляторами: универсальным и широко
используемым в промышленной автоматике ПИД регулятором и винеровским
--- оптимальным в случае линейной САУ.  Сравнение регуляторов
проводится в различных условиях на основе интегрального
(среднеквадратичная ошибка) и экстремального (максимальная ошибка)
критериев.

\subsubsection{Постановка задачи}

Имеется система управления с обратной связью и линейным объектом
управления.  Стохастические параметры уставки и помехи заданы.  Для
системы синтезированы ПИД регулятор, винеровский оптимальный регулятор
\footnote{Линейный объект должен иметь вид, допускающий существование
  физически реализуемого винеровского оптимального регулятора} и
нейросетевой квазиоптимальный регулятор, например, полученный во время
первой лабораторной работы.

Сравнение регуляторов следует проводить исходя из достигнутого ими
качества управления на достаточно длинном временном интервале.
Критериями качества управления являются максимальное по модулю
перерегулирование и среднеквадратичная ошибка.

Требуется экспериментально исследовать поведение трех видов
регуляторов в номинальных условиях, на разных видах сигнала уставки
(ступенчатом, гармоническом, стохастическом), а также в условиях,
отличных от номинальных: иные параметры объекта управления, уставки и
помехи (отсутствие, номинальный уровень, повышенный уровень помехи,
``цветная'' помеха).  Экспериментально определить зависимость качества
управления от частоты, подавая на вход системы гармоническую уставку.

\subsubsection{Варианты}

Данная работа может выполняться по вариантам заданных САУ (параметры
ОУ, уставки и помехи), а также по набору экспериментов, которые
следует провести.

\subsubsection{План работы}

Каждый пункт плана работы подразумевает независимый от других пунктов
сеанс моделирования для каждого из трех видов регуляторов.  Перед этим
необходимо подготовить сигналы уставки и помехи, а также установить
параметры объекта управления сообразно заданию.

\begin{enumerate}
\item Номинальная стохастическая уставка и помеха, номинальный объект
  управления.
\item Уставка --- меандр с длительностью ступеней, достаточной для
  завершения переходного процесса.  Помеха --- отсутствует.  Объект
  управления --- номинальный.
\item Уставка --- синусоида.  Помеха --- отсутствует.  Объект
  управления --- номинальный.  Повторить для нескольких различных
  частот уставки , включая частоту Найквиста.
\item Стохастическая уставка, отличная от номинальной с дисперсией,
  примерно совпадающей с номинальной.  Номинальная помеха.  Объект
  управления --- номинальный.
\item Стохастическая уставка, отличная от номинальной с дисперсией, в
  два раза превышающая номинальную.  Номинальная помеха.  Объект
  управления --- номинальный.
\item Номинальная стохастическая уставка.  Помеха --- белый шум с
  интенсивностью в два раза больше, чем номинальная.  Объект
  управления --- номинальный.
\item Номинальная стохастическая уставка.  Помеха --- ``цветной'' шум.
  Объект управления --- номинальный.
\item Номинальная стохастическая уставка и помеха.  Параметры объекта
  управления отличаются от номинальных.
\item Номинальная стохастическая уставка и помеха.  Вид объекта
  управления отличается от номинального.
\end{enumerate}

\subsubsection{Пример отчета}


%%%%%%%%%%%%%%%%%%%%%%%%%%
% Лабораторная работа №3 %
%%%%%%%%%%%%%%%%%%%%%%%%%%
\subsection{Нейросетевое управление нестационарным объектом}

\subsubsection{Цель работы}

Практическая работа нацелена на построение нейросетевой системы
управления нестационарным объектом.  Рассматриваются вопросы
обнаружения разладки с помощью нейросетевой модели объекта и алгоритма
куммулятивных сумм, сбора обучающих данных и адаптации
регулятора в контуре.

\subsubsection{Постановка задачи}

Рассматривается задача адаптации нейросетевого регулятора при
обнаружении изменений в поведении объекта управления.  Считаем, что
изменения происходят достаточно редко и ступенчато.  Уставка и помеха
являются стохастическими и не меняющими своих свойств.

Кроме нейросетевого регулятора имеется также нейросетевая модель
объекта управления, работающая параллельно с объектом и выдающая
предсказание его выхода.  Разность предсказания и выхода является
ошибкой идентификации.  При обнаруженном росте дисперсии этой ошибки
считается установленным факт разладки, то есть, изменения объекта
управления.  Нейросетевой регулятор и модель объекта получены в ходе
первой лабораторной работы.

Для обнаружения разладки по дисперсии ошибки идентификации применяется
алгоритм кумулятивных сумм, настраиваемый с целью обеспечить
компромисс между минимизацией среднего времени запаздывания при
обнаружении разладки и максимизацией среднего времени между ложными
тревогами.  Основными управляемыми параметрами АКС являются порог
(решающая граница) и номинальная разладка.

После обнаружения разладки активируется алгоритм сбора данных для
коррекции модели объекта управления.  Нейросеть модели объекта
управления корректируется вне контура и после завершения процесса её
обучения включается алгоритм адаптации нейросетевого регулятора в
контуре управления.  При стабилизации уменьшившейся ошибки управления
алгоритм адаптации отключается.

\subsubsection{Варианты}

Различные варианты заданий могут быть сформированы за счет различных
объектов управления, исходных нейросетевых регуляторов и параметров
уставки и помехи.

\subsubsection{План работы}

\begin{enumerate}
\item Настройка параметров АКС для обнаружения разладки.  Уровень
  номинальной разладки и порог подбираются экспериментально до
  получения подходящих значений среднего времени запаздывания и
  среднего времени между ложными тревогами.
\item Моделирование САУ с изменением параметров ОУ в заданный момент
  времени.  Диагностика факта разладки с помощью АКС.
\item Сбор данных для коррекции нейросетевой модели объекта
  управления.
\item Обучение нейросетевой модели предсказанию наблюдаемого выхода
  объекта управления.
\item Включение контура адаптации нейросетевого регулятора с помощью
  скорректированной нейросетевой модели.  Прекращение адаптации по
  достижении стабильного уровня ошибки управления.
\item Моделирование САУ с новым нейросетевым регулятором в контуре
  с целью сравнения его с исходным.
\end{enumerate}

\subsubsection{Пример отчета}
