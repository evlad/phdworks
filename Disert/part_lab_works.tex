% -*-coding: utf-8;-*-
%%%%%%%%%%%%%%%%%%%%%%%%%%%%%%%%%%%%%%%%%%%%%%%%%%%%%%%%%%%%%%%%%%%%%%%%%%%%%%
% Программный комплекс для обучения методам нейросетевого управления в
% учебном процессе

\section{Цель работы}



Разработка интерактивного программного комплекса для настройки и
моделирования традиционных и нейросетевых систем управления с обратной
связью.



\subsection{Мотивация}

Искусственные нейронные сети представляют собой относительно новый
мощный и многогранный инструмент для решения разнообразных инженерных
и научно-исследовательских задач.  Способ применения нейронных сетей
достаточно специфичен, так как он основан не на аналитическом
формализме, присущем человеческому интеллекту, а на машинном обучении
--- неявном извлечении взаимосвязей из представленных данных.  Эта
особенность отличает нейросетевые методы от традиционных, в том числе
и в задачах автоматического управления.

Аналитические методы давно и прочно укоренились в учебных курсах на
инженерных специальностях.  Для освоения аппарата этих методов
студентам предлагается большое количество разнообразных упражнений,
суть которых сводится к аналитическому решению задачи, то есть, к
выводу формальным способом решения в общем виде.  Только в том случае,
если математический аппарат не позволяет получить точное решение в
аналитической форме, используются приближенные способы решения, в том
числе, с помощью численных методов.  Однако в любом случае вид
решения, а значит и его свойства, определяются человеком.

Нейросетевые методы подразумевают участие человека только в
обеспечении процесса обучения, а собственно решение формируется
нейросетью само.  Его вид и свойства изначально неизвестны.  Более
того, после успешного завершения обучения нейронной сетью, отличной от
тривиальной, человеку практически невозможно понять, какие решения
приняла сеть и и как это повлияло на достижение поставленной цели.
Извлечение знаний из обученной нейронной сети представляет собой
отдельную проблему.

Причину этого следует искать в том, что существующая на настоящий
момент теория искусственных нейронных сетей неполна.  В частности, не
существует математически доказанных конструктивных методов,
обеспечивающих решение базовой задачи интерполяции произвольной
функции нейросетью.  Вопросы информационной ёмкости нейронной сети
решены только для небольшого класса сетей с линейными нейронами,
бесперспективность которого была показана ещё в 1960-х годах Мински
(???).  Архитектура и принципы функционирования нейронных сетей разных
типов чрезвычайно разнообразны, а математический аппарат, используемый
в нейросетевых методах решения прикладных задач, позаимствован из
разных областей математики.

Возможность применения нейронных сетей основывается только на теореме
(???) о существовании решения интерполяции произвольной функции
многослойным нелинейным персептроном.  В отсутствии теоретически
обоснованного конструктивного метода настройки нейронной сети широкое
распространение получили различные численные методы, близкие к
нелинейной оптимизации, а также разнообразные эвристические подходы.

Изложенные особенности свидетельствуют об особой важности практических
работ в учебном курсе по нейросетевым методам.  Только они дают
возможность студентам увидеть, как происходит процесс обучения и как
функционирует обученная нейронная сеть в рамках решения конкретной
прикладной задачи.  Это невозможно до конца формализовать на лекциях и
изложить в учебных пособиях.  Фактически, совокупность архитектуры НС,
метода обучения с его параметрами, а также обучающих данных дает
уникальное решение, выражающееся в наборе весовых коэффициентов НС.

Обычно для практических работ по курсу искусственных нейронных сетей
используется тот или иной универсальный (Statistica, MatLab, Octave)
или специализированный нейросетевой (Stuttgart Neural Network
Simulator, Neural Lab, Trajan) программный пакет.  Для большинства
типовых задач, решаемых с помощью НС (распознавание образов,
ассоциативная память, кластеризацияб предсказание и т.п.),
возможностей перечисленных пакетов вполне достаточно.  К ним
относятся:

\begin{itemize}
\item Задание архитектуры нейросети и метода её обучения
\item Задание обучающего множества
\item Задание параметров обучения
\item Обучение нейронной сети
\item Анализ качества работы обученной нейронной сети
\end{itemize}

Однако применение НС в задачах управления требует дополнительно
наличия многих функций, отсутствующих в пакетах нейросетевого
моделирования:

\begin{itemize}
\item Задание вида и параметров регулятора и объекта управления
\item Задание входных сигналов - уставки и помехи
\item Съем данных из различных точек контура управления для
  визуализации и обучения НС
\item Моделирование САУ
\item Сравнение и анализ качества работы САУ с различными
  компонентами, в том числе, с нейросетевыми
\end{itemize}

В универсальных пакетах (класса MatLab) реализация перечисленных
функций требует достаточно серьезного программирования как
вычислительных, так и интерактивных и графических функций.  В то же
время, получающаяся программа обладала бы ограниченным
быстродействием, так как должна быть написана на интерпретируемом
языке программирования.  Вопросы быстродействия в данном классе задач
достаточно важны, так как моделирование и обучение нейронных сетей
производится на длинных временных рядах (порядка $10^4 ... 10^6$
отсчетов).

Представляется актуальным разработать интерактивный пакет программ,
позволяющий решать задачи нейросетевого управления и сопоставлять
нейросетевые подходы с традиционными.  На базе такого пакета можно
создать курс практических занятий для студентов инженерных
специальностей, изучающих нейросетевые методы вообще и применение
нейронных сетей в системах автоматического управления в частности.

Расширение лабораторной базы учебного процесса позволит укрепить
знания, получаемые студентами на лекциях, практическим опытом решения
учебных задач.  

При определенной гибкости настроек подобный программный комплекс был
бы полезен и для исследовательских проектов, позволяя быстро проводить
моделирование проектируемой САУ и оценивать возможности использования
в ней нейросетевого управления.

\section{Постановка задачи}

При всем многообразии возможных практических работ наиболее важными
представляются следующие три:

\begin{itemize}
\item Синтез нейросетевого регулятора
\item Сравнительный анализ нейросетевого, винеровского и ПИД регуляторов
\item Управление нестационарным объектом
\end{itemize}

Рассмотрим более подробно структуру этих лабораторных работ

Совершенствование лабораторной базы.  Предлагается создать лабораторные работы.

Программный комплекс.
Задачи, возложенные на этот комплекс.

\subsection{Структура комплекса}
\subsection{Принципы взаимодействия}

\subsection{Дидактические цели}
Что должен знать и уметь студент по результатам выполнения лабораторных работ.

\subsection{План методички}
\begin{itemize}
\item Введение: цель работы
\item Теоретическая часть - взять из 2, 3, 4 глав
\item Задание на выполнение работы
\item Инструкция по проведению работы (описание программного комплекса)
\item Представление результатов
\item Контрольные вопросы
\end{itemize}

\section{Методическая база}
\subsection{Примеры отчетов - в приложении}
