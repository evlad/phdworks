% -*-coding: utf-8;-*-
%%%%%%%%%%%%%%%%%%%%%%%%%%%%%%%%%%%%%%%%%%%%%%%%%%%%%%%%%%%%%%%%%
% Введение

\paragraph{Обоснование выбора темы}

В современном мире системы автоматического управления занимают важное
место, обусловленное широким распространением сложных технических
устройств во свех сферах жизни: науке, производстве, образовании,
сфере услуг и в быту.  В идеале автоматическое регулирование
обеспечивает поддержание заданных характеристик функционирования
системы без постоянного вмешательства человека.  Удобство и
необременительность автоматически управляемых систем оборачивается
высокими требованиями к качеству, которое должен обеспечивать алгоритм
регулирования при функционировании в реальных условиях.  Наличие
большого числа факторов, влияющих на параметры технической системы в
процессе её производства и последующего применения часто усложняет
разработку алгоритма регулирования и удорожает в итоге систему и её
эксплуатацию.

Адекватным ответом на вызовы времени является разработка новых
подходов при синтезе алгоритмов автоматического регулирования,
позволяющих снизить стоимость разработки систем управления без
снижения качества или увеличить качество по сравнению с существующими
подходами.  Применение искусственных нейронных сетей в системах
автоматического управления является достаточно новым направлением,
получившим значительное развитие с конца 80-х годов XX века.
Уникальные свойства нейронных сетей, такие как возможность
аппроксимации произвольной непрерывной функции, неподверженность
``проклятию размерности'', неявное извлечение зависимостей из массива
данных, позволяют решать по-новому многие задачи автоматического
регулирования.

\paragraph{Объект и предмет исследований}

%Ссылки на корифеев

Первые области практического применения искусственных нейронных сетей
были связаны с задачами распознавания и ассоциативной
выборки~\cite{wasser92,koh80,gal74}.  Объясняется это прежде всего
простотой описания входных и выходных данных нейросети и критерия
качества функционирования.  В 80-е годы появились первые публикации,
посвященные применению нейронных сетей в задачах управления~(например,
\cite{barto83}).  В целом, учитывая назначение биологического
прототипа, искусственные нейронные сети вполне адекватны задачам
управления~\cite{bondlog97}, ведь в живом организме нервные сети
решают задачи управления: перемещение, регуляция, сохранение
равновесия.

При всем многообразии частных подходов, к настоящему времени
сформулированы основные схемы применения нейронных сетей в системах
управления~\cite{narpart92,park96,suykens96,terehov99,sigom00}.
Однако знакомство с публикациями по нейросетевому управлению оставляет
двойственное ощущение.  С одной стороны, предлагаются разнообразные
подходы, архитектуры нейросетей, методы их настройки, исследователи
рапортуют об успешном решении поставленных задач, зачастую с лучшим
качеством по сравнению с традиционными методами.  С другой стороны,
без ответа остаются вопросы о причинах выбора той или иной
нейросетевой архитектуры, её оптимальности, возможных альтернативах,
подготовке обучающих данных, степени влияния помех.  Не всегда есть
ясность в понимании причин успеха нейросети при решении конкретной
задачи.  Нет понимания относительно границ области применения --
всегда ли нейросеть лучше линейного регулятора?  Многие исследователи
сравнивают своё нейросетевое решение задачи управления с традиционным
(как правило, это ПИД регулятор), но публикации, где делается попытка
систематического сравнения линейных и нейросетевых регуляторов хотя бы
на конкретной задаче, чрезвычайно редки~\cite{khomyu96}, равно как и
критика нейросетевых подходов~\cite{warwick95,warwick96}.

% Привязать нестационарные системы и обучающий софт

Таким образом, методический базис применения нейросетевых алгоритмов в
системах управления выглядит неполным.

\paragraph{Цель исследований}

Представляется актуальным систематически рассмотреть вопросы
методического характера:
\begin{enumerate}
\item Выбор архитектуры нейросети.
\item Выбор объема обучающей выборки.
\item Сравнение нейросетевого регулятора с линейными аналогами.
\end{enumerate}

\paragraph{Используемый метод}

Поскольку нет теоретически обоснованных конструктивных подходов
применяем метод имитационного моделирования на компьютере.

\paragraph{Теоретическая значимость и практическая ценность}

Новизна.

Актуальность.

\paragraph{Основные положения, выносимые на защиту}

\paragraph{Краткий обзор глав}

%В первой главе делается обзор актуальных направлений по применению
%нейронных сетей в задачах управления.  Отмечаются 

\paragraph{Указание на наличие приложений}

\paragraph{Ссылки на собственные труды}

