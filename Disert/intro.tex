% -*-coding: utf-8;-*-
%%%%%%%%%%%%%%%%%%%%%%%%%%%%%%%%%%%%%%%%%%%%%%%%%%%%%%%%%%%%%%%%%
% Введение

\paragraph{Обоснование выбора темы}

В современном мире системы автоматического управления занимают важное
место, обусловленное широким распространением сложных технических
устройств во свех сферах жизни: науке, производстве, образовании,
сфере услуг и в быту.  В идеале автоматическое регулирование
обеспечивает поддержание заданных характеристик функционирования
системы без постоянного вмешательства человека.  Удобство и
необременительность автоматически управляемых систем оборачивается
высокими требованиями к качеству, которое должен обеспечивать алгоритм
регулирования при функционировании в реальных условиях.  Наличие
большого числа факторов, влияющих на параметры технической системы в
процессе её производства и последующего применения часто усложняет
разработку алгоритма регулирования и удорожает в итоге систему и её
эксплуатацию.

Адекватным ответом на вызовы времени является разработка новых
алгоритмов автоматического регулирования, позволяющих снизить
стоимость разработки систем управления без снижения качества или
увеличить качество по сравнению с существующими подходами.  Применение
искусственных нейронных сетей в системах автоматического управления
является достаточно новым направлением, получившим значительное
развитие с конца 80-х годов XX века.  Уникальные свойства нейронных
сетей, такие как возможность аппроксимации произвольной непрерывной
функции, неподверженность ``проклятию размерности'', неявное
извлечение зависимостей из массива данных, позволяют решать по-новому
многие задачи автоматического регулирования.

\paragraph{Объект и предмет исследований}

%Ссылки на корифеев

Первые области практического применения искусственных нейронных сетей
были связаны с задачами распознавания и ассоциативной
выборки~\cite{wasser92,koh80,gal74}.  Объясняется это прежде всего
простотой описания входных и выходных данных нейросети и критерия
качества функционирования.  В 80-е годы появились первые публикации,
посвященные применению нейронных сетей в задачах управления~(например,
\cite{barto83}).  В целом, учитывая назначение биологического
прототипа, искусственные нейронные сети вполне адекватны задачам
управления~\cite{bondlog97}, ведь в живом организме нервные сети
решают задачи управления: перемещение, регуляция, сохранение
равновесия.

При всем многообразии частных подходов, к настоящему времени
сформулированы основные схемы применения нейронных сетей в системах
управления с обратной
связью~\cite{narpart92,park96,suykens96,terehov99,sigom00}.  Однако
знакомство с публикациями по нейросетевому управлению оставляет
двойственное впечатление.  С одной стороны, предлагаются разнообразные
подходы, архитектуры нейросетей, методы их настройки, исследователи
рапортуют об успешном решении поставленных задач, зачастую с лучшим
качеством по сравнению с традиционными методами.  С другой стороны,
без ответа остаются вопросы о причинах выбора той или иной
нейросетевой архитектуры, её оптимальности, возможных альтернативах,
подготовке обучающих данных, степени влияния помех.  Не всегда есть
ясность в понимании причин успеха нейросети при решении конкретной
задачи.  Нет понимания относительно границ области применения --
всегда ли нейросеть лучше линейного регулятора?  Многие исследователи
сравнивают своё нейросетевое решение задачи управления с традиционным
(как правило, это ПИД регулятор), но публикации, где делается попытка
систематического сравнения линейных и нейросетевых регуляторов хотя бы
на конкретной задаче, чрезвычайно редки~\cite{khomyu96}, равно как и
критика нейросетевых подходов~\cite{warwick95,warwick96}.

% Привязать нестационарные системы и обучающий софт

Некоторые вопросы, касающиеся выбора архитектуры, обучающего множества
и параметров обучения, могла бы снять проработанная конструктивная
теория нейронных сетей, однако к настоящему времени её нет.
Располагаемые теоретические наработки: теорема Колмогорова о
существовании решения задачи аппроксимации непрерывной
функции~\cite{kolmog57} применительно к искусственным нейросетям,
расчет ёмкости нейронной сети в задачах классификации с помощью
измерения Вапника-Червоненкиса~\cite{haykin2008} и аппарат многомерной
оптимизации, используемый при обучении с учителем, --- не дают
ответов, имеющих практическую ценность.  Таким образом, методический
базис применения нейросетевых алгоритмов в системах управления
представляется неполным.  Без него синтез нейронных сетей становится
эмпирическим исследовательским процессом, плохо вписывающимся в
инженерную и производственную практику.

\paragraph{Цель исследований}

Представляется актуальным систематически рассмотреть вопросы
методического характера, возникающие при применении нейронных сетей в
системах управления с обратной связью:
\begin{enumerate}
\item Выбор архитектуры нейросети.
\item Выбор объема и вида обучающей выборки.
\item Управление процессом обучения.
\item Сравнение нейросетевого регулятора с линейными аналогами с целью
  выяснения его характерных свойств.
\end{enumerate}

В качестве типовых задач предлагается рассмотреть следующие, наиболее
общие и часто встречающиеся в публикациях:
\begin{enumerate}
\item Синтез нейросетевого аналога ПИД регулятора.
\item Синтез нейросетевого регулятора с помощью нейросетевой модели
  объекта управления.
\item Адаптация нейросетевого регулятора для управления нестационарным
  объектом.
\end{enumerate}

При решении задач синтеза представляется целесообразным принимать во
внимание особенности реальных систем управления: наличие помех,
возможное отклонение фактических параметров объекта от проектных,
ограничения на пробные сигналы, невозможность произвольных
экспериментов с объектом управления в силу физических, конструкционных
и эксплуатационных ограничений.  Кроме того, важно предусмотреть
должный уровень автоматизации при реализации поддержки методических
рекомендаций.  Это повысит их практическую полезность.

\paragraph{Используемый метод}

В силу упоминавшегося выше отсутствия конструктивной теории
искусственных нейронных сетей применяется методов имитационного
моделирования на компьютере.  Для обеспечения общности выводов
вычислительные эксперименты проводятся с разными видами и параметрами
регулятора, объекта управления, уставки и помехи.

\paragraph{Теоретическая значимость и практическая ценность}

Впервые представлена методика синтеза нейросетевого аналога ПИД
регулятора, содержащая рекомендации относительно выбора количества и
номенклатуры входов нейросети регулятора, количества слоёв сети и
определения объема обучающей выборки.  Также впервые исследован вопрос
влияния вида пробных сигналов на качество обучения по полученной
выборке и дано объяснено наблюдаемым эффектам.

Разработана новая методика синтеза нейросетевого оптимального
регулятора для замены имеющегося в контуре управления с целью
улучшения качества в терминах среднеквадратической ошибки управления.
Методика ориентирована на использование в реальных системах
управления.  В рамках методики систематически рассмотрены вопросы
выбора архитектуры нейросетевой модели объекта управления,
формирования обучающей выборки, а также сформулирован критерий
останова процедуры обучения нейросетевого оптимального регулятора в
контуре управления.

Впервые проведены сравнительные эксперименты винеровского оптимального
и нейросетевого оптимального регуляторов.  Выявлены характерные
особенности нейросетевого регулятора, отличающие его от линейных, и
дано объяснение этих свойств.  Отмечены как положительные, так
отрицательные стороны применения нейросетевых регуляторов в системах
управления.

Предложена новая методика нейросетевого автоматического управления
нестационарным объектом, использующая постоянно действующую
нейросетевую модель и алгоритм кумулятивных сумм (АКС) для определения
момента изменения параметров объекта управления.  Исследованы вопросы
подбора параметров АКС для получения желаемых характеристик системы
упрвавления.

Разработанные методы успешно опробованы в вычислительных экспериментах
на линейных и нелинейных объектах, в том числе, при наличии помехи, а
также на реальном объекте --- мобильном колёсном роботе.

Разработан оригинальный программный комплекс для синтеза и
исследования традиционных и нейросетевых алгоритмов управления.
Комплекс адаптирован для использования в учебном процессе в рамках
курса изучения нейронных сетей.  На основе программного комплекса
подготовлены лабораторные работы.

\paragraph{Основные положения, выносимые на защиту}

\begin{enumerate}
\item Методика синтеза нейросетевого аналога ПИД регулятора.
\item Методика синтеза нейросетевого оптимального регулятора.
\item Методика нейросетевого управления нестационарным объектом.
\end{enumerate}

\paragraph{Краткий обзор глав}

% Глава 1

В первой главе делается обзор прикладных областей, в которых активно
исследуются вопросы применения нейронных сетей в системах управления.
Дается краткий анализ причин актуальности этой работы.

Рассматриваются основные архитектуры нейронных сетей, применяемые в
системах управления.  Описывается их устройство и приводится их
классификация по признаку способа реализации динамических свойств.
Перечисляются базовые и усовершенствованные алгоритмы обучения.
Формулируются основные проблемы, возникающие при проектировании
нейросети для решения конкретной прикладной задачи.

На основе анализа литературы перечисляются основные способы применения
нейронных сетей в системах управления.  Подробно рассматриваются и
анализируются различные подходы к синтезу нейросетевого регулятора.
Описываются основные схемы обучения, применяемые исследователями, и их
свойства.  Далее делается обзор методов нейросетевой идентификации, а
также приводится информация по классическим методам.  Приводятся схемы
реализации нейросетевых моделей и отмечаются характерные особенности
при их синтезе и эксплуатации.  В отдельную группу выделены прочие
способы использования потенциала нейронных сетей.  Сюда отнесены
гибридные регуляторы, нейросети--настройщики регуляторов иных типов и
алгоритмы нейросетевой фильтрации.

Подчеркивается важность проведения систематического сопоставления
традиционных линейных и нейросетевых регуляторов, а также выявления
отличительных свойств обоих типов регуляторов.  Анализируются
публикации по данной тематике.  Отмечается целесообразность
сопоставления как с наиболее распространенным ПИД регулятором по
причине высокой практической ценности такого исследования, так и с
винеровским оптимальным регулятором по причине эквивалентности
постановки задачи синтеза, что делает сопоставление максимально
корректным.

Рассматриваются основые задачи, возникающие при изучении нейронных
сетей в системах управления.  Отмечается неполнота их реализации в
основных пакетах универсального и нейросетевого моделирования.
Отдельно рассматриваются online-ресурсы, используемые для демонстрации
нейросетевых алгоритмов.  Отмечаются их недостатки с точки зрения
исследователя и преподавателя высшей школы.  Делается вывод об
актуальности разработки программного пакета, совмещающего в себе
возможности моделирования САУ и обучения нейронных сетей для решения
задач управления с целью использования в учебном процессе.

% Глава 2

Во второй главе формулируется задача замены ПИД регулятора на
нейросетевой с позиции аппроксимации функции ПИД регулятора нейросетью
вне контура управления по критерию минимизации среднеквадратической
ошибки имитации.  Определяются основные этапы методики синтеза
нейросетевого регулятора (НС--Р), среди которых выбор архитектуры
нейросети, сбор обучающих данных, обучение, проверка качества имитации
и функционирования.

Далее рассматривается вопрос выбора архитектуры нейросети регулятора.
Под архитектурой понимается совокупность набора входов для реализации
динамических свойств, количества слоев сети и распределения нейронов в
них.  Последовательно рассматриваются варианты компонентов архитектуры
и в рамках имитационных экспериментов выявляются наиболее приемлемые
из них.

Проводится исследование о влиянии вида пробного сигнала, используемого
для получения обучающей выборки, на качество имитации НС--Р.
Последующий анализ позволяет сформулировать основной критерий,
обеспечивающий качество результирующего нейросетевого регулятора ---
степень равномерности распределения облака обучающих точек в
многомерном пространстве области определения функции аппроксимации.
Достижение минимальной плотности покрытия области определения дает
необходимую длину обучающей выборки.

Приводится алгоритм пакетного обучения НС--Р методом обратного
распространения ошибки.  Обсуждаются вопросы выбора коэффициента
скорости обучения и критерия останова, а также контроля за обобщающей
способностью сети в процессе обучения.  Приводятся эвристические
решения этих вопросов, успешно используемые автором в вычислительных
экспериментах.  Отмечается неэквивалентность качества имитации
исходного регулятора вне контура управления и в нём.

В качестве примера сформулированной методики синтеза рассматривается
задача управления температурой в химическом реакторе непрерывного
действия с мешалкой.  Объект управления является существенно
нелинейным и управляется ПИД регулятором.  Последовательно применяя
рекомендации методики синтезируется нейросетевой регулятор,
обеспечивающий качество управления не хуже исходного.  Отмечается
важность выбора подходящей архитектуры НС--Р.  Анализируются свойства
полученного нейросетевого регулятора по сравнению с ПИД.

% Глава 3

В третьей главе


% Глава 4
% Глава 5
% Глава 6

\paragraph{Указание на наличие приложений}

\paragraph{Ссылки на собственные труды}
