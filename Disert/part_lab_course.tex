% -*-coding: utf-8;-*-
%\section{Применение пакета в курсе лабораторных работ}

На базе разработанного программного пакета моделирования и
конструирования традиционных и нейросетевых систем управления был
разработан курс лабораторных работ.  Этот курс демонстрирует, с одной
стороны, возможности применения нейронных сетей для решения различных
задач автоматического управления, с другой стороны, возможности
программного пакета для моделирования САУ и обучения нейронных сетей.

%\subsection{Интеграция в учебный процесс}

Разрабатываемый курс лабораторных работ предназначен для расширения
существующего курса, посвященного применению искусственных нейронных
сетей в целом.  В рамках имеющихся ограничений по времени на
нейросетевые системы управления выделено три практических занятия
длительностью 3--4 часа каждое.  Представляется оптимальным
скомпоновать учебный материал в занятия по следующим темам:

\begin{itemize}
\item Синтез нейросетевого оптимального регулятора.
\item Сравнительный анализ нейросетевого, винеровского и ПИД регуляторов.
\item Нейросетевое управление нестационарным объектом.
\end{itemize}

Первая лабораторная работа посвящается методике замены регулятора в
существующей системе управления на нейросетевой.  Для этого необходимо
провести моделирование и собрать данные для начального обучения
нейросетевого регулятора и модели объекта.  После этого требуется
настроить нейронные сети регулятора и модели объекта.  Предварительно
настроенный регулятор включается в САУ на место исходного и с помощью
модели объекта в процессе работы подстраивается для минимизации ошибки
управления.

В рамках второй работы требуется сравнить настроенный нейросетевой
регулятор с винеровским оптимальным и ПИД регулятором в различных
условиях, как номинальных (используемых при настройке), так и
отличающихся от них.  В частности, следует провести моделирование с
разными типами сигналов уставки и интенсивностью помехи,
экспериментально исследовать зависимость качества управления от
частоты, определить, насколько чувствителен регулятор к точности
задания параметров объекта управления.

Третья лабораторная работа посвящена задаче нейросетевого управления
нестационарным объектом.  Она включает настройку системы обнаружения
разладки с помощью нейросетевой модели и алгоритма кумулятивных сумм
(АКС), сбор данных и подстройку с их помощью модели объекта
управления, а также подстройку нейросетевого регулятора в контуре.

Тематика лабораторных работ достаточно разнообразна.  В частности,
рассматриваются как конструкторские (синтез элементов САУ, подключение
АКС), так и исследовательские задачи (выбор оптимальной архитектуры
нейросети, сравнение различных регуляторов).  Сопоставление с
линейными регуляторами, в том числе, с винеровским оптимальным,
связывает материал курса с классической теорией автоматического
управления.  Использование АКС для обнаружения разладки демонстрирует
возможность совместного использования нейросетевых и традиционных
алгоритмов.

%\subsection{Дидактические аспекты}

Успешное выполнение практической работы требует от студента следующих
видов деятельности:

\begin{itemize}

\item изучение теоретических основ нейронных сетей;

\item освоение программного комплекса;

\item подготовку отчета.

\end{itemize}

Материал, вошедший в лабораторные работы, подразумевает, что помимо
теоретических знаний лекционной части курса студент обладает знаниями
по математической статистике, теории функций комплексного переменного
и линейной теории автоматического управления (для детерминированных и
стохастических сигналов).

В результате выполнения практических работ студент должен изучить:
\begin{itemize}

\item Архитектуру нейронных сетей прямого распространения.

\item Способы реализации нейронными сетями динамических свойств
  элементов системы автоматического управления ($\{{e_k,r_k}\}$,
  $\{e_k,e_{k-1},...\}$, $\{u_k,u_{k-1},...y_k,y_{k-1},...\}$).

\item Влияние выбранного критерия качества управления, используемого
  при настройке регулятора, на функционирование САУ в условиях,
  приближенных к реальным (помеха, неточная параметризация сигналов и
  объекта управления, нестационарность).

\item Метод обратного распространения ошибки и его применение при
  синтезе элементов нейросетевой системы управления (начальная
  настройка и подстройка в контуре нейросетевого регулятора, настройка
  нейросетевой модели объекта управления).

\item Роль нейросетевой модели объекта при синтезе нейросетевого
  регулятора и при обнаружении разладки.

\item Влияние обучающих и контрольных данных на свойства обученной
  нейронной сети (область определения и область значений нейросети).

\item Основные свойства нейросетевых регуляторов и их отличия от
  линейных регуляторов (частотная характеристика, зависимость от
  уровня сигналов, устойчивость, адаптированность к нелинейным
  объектам и пр.).

\item Методы нейросетевого управления нестационарным объектом и их
  особенности (быстродействие, ресурсоемкость, устойчивость).

\end{itemize}
